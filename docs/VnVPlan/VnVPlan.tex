\documentclass[12pt, titlepage]{article}

\usepackage{booktabs}
\usepackage{tabularx}
\usepackage{hyperref}
\hypersetup{
    colorlinks,
    citecolor=blue,
    filecolor=black,
    linkcolor=red,
    urlcolor=blue
}
\usepackage[round]{natbib}
\usepackage{amssymb}


%% Comments

\usepackage{color}

%\newif\ifcomments\commentstrue %displays comments
\newif\ifcomments\commentsfalse %so that comments do not display

\ifcomments
\newcommand{\authornote}[3]{\textcolor{#1}{[#3 ---#2]}}
\newcommand{\todo}[1]{\textcolor{red}{[TODO: #1]}}
\else
\newcommand{\authornote}[3]{}
\newcommand{\todo}[1]{}
\fi

\newcommand{\wss}[1]{\authornote{blue}{SS}{#1}} 
\newcommand{\plt}[1]{\authornote{magenta}{TPLT}{#1}} %For explanation of the template
\newcommand{\an}[1]{\authornote{cyan}{Author}{#1}}

%% Common Parts

\newcommand{\progname}{Inverted Pendulum Control Systems} % PUT YOUR PROGRAM NAME HERE
\newcommand{\authname}{Morteza Mirzaei} % AUTHOR NAMES                  

\usepackage{hyperref}
    \hypersetup{colorlinks=true, linkcolor=blue, citecolor=blue, filecolor=blue,
                urlcolor=blue, unicode=false}
    \urlstyle{same}
                                


\begin{document}

\title{System Verification and Validation Plan\\ for\\ \progname{}\\(IPCS)} 
\author{\authname}
\date{\today}
	
\maketitle

\pagenumbering{roman}

\section*{Revision History}

\begin{tabularx}{\textwidth}{p{3cm}p{2cm}X}
\toprule {\bf Date} & {\bf Version} & {\bf Notes}\\
\midrule
2024-02-18 & 1.0 & Initial Release.\\
\bottomrule
\end{tabularx}

~\\
\wss{The intention of the VnV plan is to increase confidence in the software.
However, this does not mean listing every verification and validation technique
that has ever been devised.  The VnV plan should also be a \textbf{feasible}
plan. Execution of the plan should be possible with the time and team available.
If the full plan cannot be completed during the time available, it can either be
modified to ``fake it'', or a better solution is to add a section describing
what work has been completed and what work is still planned for the future.}

\wss{The VnV plan is typically started after the requirements stage, but before
the design stage.  This means that the sections related to unit testing cannot
initially be completed.  The sections will be filled in after the design stage
is complete.  the final version of the VnV plan should have all sections filled
in.}

\newpage

\tableofcontents

\listoftables
\wss{Remove this section if it isn't needed}

\listoffigures
\wss{Remove this section if it isn't needed}

\newpage

\section{Symbols, Abbreviations, and Acronyms}

For complete symbols used within the system, please refer the section 1 in 
\href{https://github.com/mirzaim/ipcs/blob/main/docs/SRS/SRS.pdf}{SRS} document.
\hyperref[table_abb]{Table \ref*{table_abb}} defines some
abbreviations and acronyms used in this document.

% \renewcommand{\arraystretch}{1.2}
% \begin{tabular}{l l} 
%   \toprule		
%   \textbf{symbol} & \textbf{description}\\
%   \midrule 
%   SRS & Software Requirement Specification\\
%   VnV & Verification and Validation \\ 
%   MIS & Module Interface Specification \\
%   IPCS & Inverted Pendulum Control Systems \\ 
%   TC & Test Case \\
%   \bottomrule \label{table_abb}
% \end{tabular}\\

\begin{table}[!h]
  \centering
  \caption{Table of Abbreviations and Acronyms.}
  \renewcommand{\arraystretch}{1.2}
  \begin{tabular}{l l} 
    \toprule		
    \textbf{symbol} & \textbf{description}\\
    \midrule 
    SRS & Software Requirement Specification\\
    VnV & Verification and Validation \\ 
    MIS & Module Interface Specification \\
    IPCS & Inverted Pendulum Control Systems \\ 
    TC & Test Case \\
    \bottomrule 
  \end{tabular}\\
  \label{table_abb}
\end{table}

\wss{symbols, abbreviations, or acronyms --- you can simply reference the SRS
  \citep{SRS} tables, if appropriate}

\wss{Remove this section if it isn't needed}

\newpage

\pagenumbering{arabic}

This document outlines the framework for Verification and Validation (VnV), 
delineating the objectives, scope, and methodologies for confirming 
the correctness and suitability of the software solution. Verification entails 
scrutinizing whether the software aligns with predetermined requirements and 
design parameters, whereas validation ensures it meets user expectations and 
demands effectively. Through this VnV framework, the team can uphold the integrity, 
dependability, and utility of the software, while also identifying and 
rectifying any flaws or discrepancies before deployment or release. 
\wss{provide an introductory blurb and roadmap of the
  Verification and Validation plan}

\section{General Information}
This section will provide the background and objectives for this document.

\subsection{Summary}

This document presents a verification and validation plan for \progname{}, 
which aims to simulate, control, and visualize an inverted pendulum on a cart.

\wss{Say what software is being tested.  Give its name and a brief overview of
  its general functions.}

\subsection{Objectives}

The objectives outlined in this Verification and Validation plan are as follows:

\begin{itemize}
    \item Verify the software's capability to precisely simulate an 
          inverted pendulum.
    \item Verify the proper functioning of the control system to sustain
          the pendulum in an upright position.
    \item Confirm the correctness of the visualization.
    \item Confirm the userfriendliness and ease of understanding of the 
          visualization.
\end{itemize}

Performance testing and stress testing are outside of the scope of this project.

\wss{State what is intended to be accomplished.  The objective will be around
  the qualities that are most important for your project.  You might have
  something like: ``build confidence in the software correctness,''
  ``demonstrate adequate usability.'' etc.  You won't list all of the qualities,
  just those that are most important.}

\wss{You should also list the objectives that are out of scope.  You don't have 
the resources to do everything, so what will you be leaving out.  For instance, 
if you are not going to verify the quality of usability, state this.  It is also 
worthwhile to justify why the objectives are left out.}

\wss{The objectives are important because they highlight that you are aware of 
limitations in your resources for verification and validation.  You can't do everything, 
so what are you going to prioritize?  As an example, if your system depends on an 
external library, you can explicitly state that you will assume that external library 
has already been verified by its implementation team.}

\subsection{Relevant Documentation}

Refer to the 
\href{https://github.com/mirzaim/ipcs/blob/main/docs/SRS/SRS.pdf}{Software 
Requirements Specification} for comprehensive information regarding 
the objectives, requirements, assumptions, and underlying 
principles of the software. You could also refer to the
\href{https://github.com/mirzaim/ipcs/blob/main/docs/Design/SoftArchitecture/MG.pdf}{Module Guide} and 
\href{https://github.com/mirzaim/ipcs/blob/main/docs/Design/SoftDetailedDes/MIS.pdf}{Module Interface Specification}
for detailed information on the software's design and architecture.
The final report of running the test which described in this document, 
will be shown in \href{https://github.com/mirzaim/ipcs/blob/main/docs/VnVReport/VnVReport.pdf}{VnV Report}.

\wss{Reference relevant documentation.  This will definitely include your SRS
  and your other project documents (design documents, like MG, MIS, etc).  You
  can include these even before they are written, since by the time the project
  is done, they will be written.}

% \citet{SRS}

\wss{Don't just list the other documents.  You should explain why they are relevant and 
how they relate to your VnV efforts.}

\section{Plan}

This section will outline the strategy for validating the documentation 
and software for \progname{}. The main components to be validated 
include the Software Requirements Specification (SRS), design, 
Verification and Validation (VnV) plan, and implementation.


\wss{Introduce this section.   You can provide a roadmap of the sections to
  come.}

\subsection{Verification and Validation Team}

The team members and their responsibilities are specified in \hyperref[team]{Table \ref*{team}}.

\begin{table}[!h]
  \caption{Verification and Validation Team} \label{team}
  \vspace*{3mm}
  \begin{tabular}{|p{0.27\textwidth}|p{0.18\textwidth}|p{0.30\textwidth}|p{0.25\textwidth}|}
  \hline
  \textbf{Name} & \textbf{Document} & \textbf{Role} & \textbf{Description} \\
  \hline
  Morteza Mirzaei & All & Author & Create and manage all the documents, create the VnV plan, perform the VnV testing, verify the implementation. \\  
  \hline
  Dr. Spencer Smith & All & Instructor/Reviewer & Review the documents, design and documentation style. \\ 
  \hline
  Seyed Ali Mousavi & All & Domain Expert Reviewer & Review all the documents. \\  
  \hline
  Al Jubair Hossain & SRS & Secondary Reviewer & Review the SRS document \\
  \hline
  Hunter Ceranic & VnV Plan & Secondary Reviewer & Review the VnV plan. \\ 
  \hline 
  Adrian Sochaniwsky & MG + MIS & Secondary Reviewer & Review the MG and MIS document. \\
  \hline 
  \end{tabular}
\end{table}

\newpage

\wss{Your teammates.  Maybe your supervisor.
  You should do more than list names.  You should say what each person's role is
  for the project's verification.  A table is a good way to summarize this information.}

\subsection{SRS Verification Plan}

The SRS document will be verified in the following way:

\begin{enumerate}
\item Initial review will be performed by the assigned members
      (Dr. Spencer Smith, Seyed Ali Mousavi, and Al Jubair Hossain).
      For this, a manual review will be conducted using the provided 
      \href{https://github.com/mirzaim/ipcs/blob/main/docs/Checklists/SRS-Checklist.pdf}{SRS Checklist} 
      and following additional checklist gather from 
      \href{https://www.cs.toronto.edu/~sme/CSC340F/2005/assignments/inspections/reqts_checklist.pdf}{here}.
      \begin{itemize}\renewcommand{\labelitemi}{\scriptsize$\square$}
        \item Are all requirements actually requirements, not design or implementation solutions?
        \item Is each requirement uniquely and correctly identified?
        \item Is each requirement verifiable by testing, demonstration, review, or analysis?
        \item Do any requirements conflict with or duplicate other requirements?
        \item Is each requirement in scope for the project?
        \item Are all requirements written at a consistent and appropriate level of detail?
        \item Do the requirements provide an adequate basis for design?
      \end{itemize}
\item The reviewers will provide feedback to the author by creating an issue on GitHub,
      referencing the checklist mentioned above.
\item Morteza Mirzaei, the author, is responsible for addressing the issues.
\end{enumerate}


\wss{List any approaches you intend to use for SRS verification.  This may include
  ad hoc feedback from reviewers, like your classmates, or you may plan for 
  something more rigorous/systematic.}

\wss{Maybe create an SRS checklist?}

\subsection{Design Verification Plan} \label{sebsec_design_verification_plan}

Initial review will be performed by the assigned members 
(Dr. Spencer Smith, Seyed Ali Mousavi, and Adrian Sochaniwsky). 
For this, a manual review will be conducted using the following checklist.

\begin{itemize}\renewcommand{\labelitemi}{\scriptsize$\square$}
  \item Does the design meet all specified requirements outlined in the design specification document?
  \item Are all interfaces between design components specified clearly?
  \item Are all components of the design testable separately?
  \item Is the design scalable to accommodate future growth or changes in requirements?
  \item Are the design components separated as much as possible?
  \item Does each design component have a logical task?
\end{itemize}
\wss{Plans for design verification}

\wss{The review will include reviews by your classmates}

\wss{Create a checklists?}

\subsection{Verification and Validation Plan Verification Plan}
The designated members (Dr. Spencer Smith, Seyed Ali Mousavi, and Hunter Ceranic) 
will conduct the initial review. This review will entail 
a manual assessment utilizing the following checklist.

\begin{itemize}\renewcommand{\labelitemi}{\scriptsize$\square$}
  \item Does the VnV plan verify all the functional and 
        non-functional requirements?
  \item Are all input and output pairs for test cases correct?
  \item Are roles and responsibilities clearly defined for individuals 
        or teams involved in verification and validation activities?
\end{itemize}

\wss{The verification and validation plan is an artifact that should also be
verified.  Techniques for this include review and mutation testing.}

\wss{The review will include reviews by your classmates}

\wss{Create a checklists?}

\subsection{Implementation Verification Plan}
\progname{} will be developed using Python programming language. 
For static testing, we will use \href{https://flake8.pycqa.org/}{Flake8}. 
Additionally, we conduct code 
walkthroughs with domain expert (Ali Mousavi). For dynamic testing, 
we employ system and unit level testing, as explained 
in detail in sections 
\hyperref[system_test]{Section \ref*{system_test}} 
and \hyperref[unit_test]{Section \ref*{unit_test}}.

\wss{You should at least point to the tests listed in this document and the unit
  testing plan.}

\wss{In this section you would also give any details of any plans for static
  verification of the implementation.  Potential techniques include code
  walkthroughs, code inspection, static analyzers, etc.}

\subsection{Automated Testing and Verification Tools}

\begin{itemize}
  \item \href{https://docs.pytest.org/}{Pytest}: A unit testing framework for Python that allows 
        for easy and scalable testing of Python code.
  \item \href{https://flake8.pycqa.org/}{Flake8}: A static code analysis tool for 
        Python that checks code against coding style (PEP 8), detects syntax errors (missing indent error), 
        or problematic constructs (cyclomatic complexity).
  \item \href{https://www.docker.com/}{Docker}: A mechanism used to containerize applications, 
        ensuring their installability and understandability throughout the project.
  \item \href{https://github.com/}{GitHub}: This platform will be used for 
        version control and collaboration.
  \item \href{https://docs.github.com/en/actions}{GitHub CI workflow}: This will 
        automate regression tests and checks that \progname{} builds are passing before 
        code is merged into a protected branch. Every push or pull request on the 
        main branch triggers the test. Additionally, after every update of 
        the main branch, the CI workflow publishes the new version of 
        the Docker container to \href{https://hub.docker.com/}{Docker Hub}.

\end{itemize}


\wss{What tools are you using for automated testing.  Likely a unit testing
  framework and maybe a profiling tool, like ValGrind.  Other possible tools
  include a static analyzer, make, continuous integration tools, test coverage
  tools, etc.  Explain your plans for summarizing code coverage metrics.
  Linters are another important class of tools.  For the programming language
  you select, you should look at the available linters.  There may also be tools
  that verify that coding standards have been respected, like flake9 for
  Python.}

\wss{If you have already done this in the development plan, you can point to
that document.}

\wss{The details of this section will likely evolve as you get closer to the
  implementation.}

\subsection{Software Validation Plan}

No plan for software validation is currently in place.
Software validation plan is beyond the scope for \progname{}
as we do not have enough experimental data 
for the validation of the system behavior.

\wss{If there is any external data that can be used for validation, you should
  point to it here.  If there are no plans for validation, you should state that
  here.}

\wss{You might want to use review sessions with the stakeholder to check that
the requirements document captures the right requirements.  Maybe task based
inspection?}

\wss{For those capstone teams with an external supervisor, the Rev 0 demo should 
be used as an opportunity to validate the requirements.  You should plan on 
demonstrating your project to your supervisor shortly after the scheduled Rev 0 demo.  
The feedback from your supervisor will be very useful for improving your project.}

\wss{For teams without an external supervisor, user testing can serve the same purpose 
as a Rev 0 demo for the supervisor.}

\wss{This section might reference back to the SRS verification section.}

\section{System Test Description} \label{system_test}

This section will define the tests to ensure IP Simulator meets 
the functional requirements seen in section 5 of the SRS document 
for simulating inverted pendulum system. The subsections combine 
several requirements that are separated based on common ideas.
	
\subsection{Tests for Functional Requirements}

This section contains the system test cases for the functional 
requirements which are described in the SRS.

\wss{Subsets of the tests may be in related, so this section is divided into
  different areas.  If there are no identifiable subsets for the tests, this
  level of document structure can be removed.}

\wss{Include a blurb here to explain why the subsections below
  cover the requirements.  References to the SRS would be good here.}

\subsubsection{Inputs and Outputs}

This part verifies the input and output restrictions outlined in R1 for \progname{}.
To review these constraints, refer to the ``Input Data Constraints" section in the SRS.
Tests are designed for both normal cases and edge and boundary cases, as outlined 
in \hyperref[table:input_constraints]{Table \ref*{table:input_constraints}}.

\wss{It would be nice to have a blurb here to explain why the subsections below
  cover the requirements.  References to the SRS would be good here.  If a section
  covers tests for input constraints, you should reference the data constraints
  table in the SRS.}

\begin{table}[ht]
\centering
\caption{Input validation test cases} \label{table:input_constraints}
\vspace*{2mm}
 \begin{tabular}{|c c c c c c|c|c|} 
 \hline
 \multicolumn{6}{|c|}{Input}                     & \multicolumn{2}{c|}{Output}            \\ \hline 
  $F$ & $x$ & $\theta$         & $m$ & $M$ & $l$ & Valid  &  Description                  \\ \hline
  4   & 0   & $\pi$            & 16  & 80  &  1  & Yes    &  -                            \\ \hline
  10  & 4   & $\frac{3\pi}{4}$ & 4   & 200 &  20 & No     &  $M$ is bigger than $M_{max}$ \\ \hline
  20  & 0   & $\pi$            & 6   & 20  &  30 & No     &  $l$ is bigger than $l_{max}$ \\ \hline
  -3  & -5  & $\pi$            & -5  & 30  &  5  & No     &  negetive value for $m$       \\ \hline
\end{tabular}
\end{table}  

\paragraph{Inputs and Outputs test}

\begin{enumerate}

\item{test-inout-1\\}

Control: Automatic
					
Initial State: N/A.
					
Input: Specified in the \hyperref[table:input_constraints]{Table \ref*{table:input_constraints}}.
					
Output: Specified in the \hyperref[table:input_constraints]{Table \ref*{table:input_constraints}}.
% \wss{The expected result for the given inputs}

Test Case Derivation: Directly from the requirement.
% \wss{Justify the expected value given in the Output field}
					
How test will be performed: 
At every stage, I input the data and assert the results by a testing framework.

\end{enumerate}

\subsubsection{Simulator} \label{func_test_simulator}

This part tests the world simulator responsible for executing 
the inverted pendulum physics. It assesses the requirements 
R2 and R3 for \progname{}. Essentially, it verifies that given the initial 
position of the pendulum and a constant force acting on the cart, 
the location of the pendulum is determined after a specified duration of time.

\begin{table}[ht]
\centering
\caption{
  Output Validation Test Cases. The time step ($dt$) is 0.1 seconds.
  The results are calculated after 5 seconds.
} \label{table:input_output}
\vspace*{2mm}
 \begin{tabular}{|c c c c c c|c c|} 
 \hline
 \multicolumn{6}{|c|}{Input} & \multicolumn{2}{c|}{Output}\\ \hline 
  $F$ & $x$ & $\theta$         & $m$ & $M$ & $l$ & $x$    & $\theta$ \\ \hline
  4   & 0   & $\pi$            & 16  & 80  &  1  & 0.535  & 4.597    \\ \hline
  10  & 4   & $\frac{3\pi}{4}$ & 4   & 50  &  20 & 6.368  & 4.005    \\ \hline
  20  & 0   & $\pi$            & 6   & 20  &  4  & 9.897  & 3.869    \\ \hline
  -3  & -5  & $\pi$            & 4   & 30  &  5  & -6.130 & 4.782    \\ \hline
\end{tabular}

\end{table}

\wss{It would be nice to have a blurb here to explain why the subsections below
  cover the requirements.  References to the SRS would be good here.  If a section
  covers tests for input constraints, you should reference the data constraints
  table in the SRS.}
		
\paragraph{Simulator}

\begin{enumerate}

\item{test-sim-1\\}

Control: Automatic
					
Initial State: Specified in the \hyperref[table:input_output]{Table \ref*{table:input_output}}.
					
Input: Specified in the \hyperref[table:input_output]{Table \ref*{table:input_output}}.
					
Output: Specified in the \hyperref[table:input_output]{Table \ref*{table:input_output}}.
% \wss{The expected result for the given inputs}

Test Case Derivation: Solving the ODE's provided in the SRS with numberical analysis tools.
% \wss{Justify the expected value given in the Output field}
					
How test will be performed: 
At every stage, I input the data and assert the results by a testing framework.

\end{enumerate}

\subsubsection{Control System}

This part tests the control system responsible for keeping 
the inverted pendulum upright. It assesses the requirements 
(R4) for \progname{}. Essentially, it verifies that after a 
reasonable amount of time, the pendulum's angular position 
remains relatively stable in an upward position.

\wss{It would be nice to have a blurb here to explain why the subsections below
  cover the requirements.  References to the SRS would be good here.  If a section
  covers tests for input constraints, you should reference the data constraints
  table in the SRS.}
		
\paragraph{Control System}

\begin{enumerate}

\item{test-control-1\\}

Control: Automatic
					
Initial State: Specified in the \hyperref[table:input_output]{Table \ref*{table:input_output}}.
					
Input: Specified in the \hyperref[table:input_output]{Table \ref*{table:input_output}}.
					
Output: $\theta$ should be between $-\frac{\pi}{4}$ and $\frac{\pi}{4}$ after 30 seconds.
% \wss{The expected result for the given inputs}

Test Case Derivation: Directly from the requirement.
% \wss{Justify the expected value given in the Output field}
					
How test will be performed: 
input the data, and after 30 seconds, I will check if the 
value of $\theta$ remains within the specified range for 1 minute.

\end{enumerate}

\subsubsection{Visualization} \label{func_test_visualization}

This part verifies Requirement 5 (R5), ensuring that the visualization 
and graphical user interface (GUI) of the software meet the necessary standards.

\wss{It would be nice to have a blurb here to explain why the subsections below
  cover the requirements.  References to the SRS would be good here.  If a section
  covers tests for input constraints, you should reference the data constraints
  table in the SRS.}
		
\paragraph{Visualization}

\begin{enumerate}

\item{test-vis-1\\}

Control: Manual
					
Initial State: Specified in the \hyperref[table:input_output]{Table \ref*{table:input_output}}.
					
Input: Specified in the \hyperref[table:input_output]{Table \ref*{table:input_output}}.
					
Output: Visualization of the pendulum at different states.
% \wss{The expected result for the given inputs}

Test Case Derivation: Directly from the requirement.
% \wss{Justify the expected value given in the Output field}
					
How test will be performed: First, Capture screenshots of the visualization 
at different states of the pendulum. Then, compare each screenshot with
the expected state of the pendulum.

\item{test-vis-2\\}

Type: Manual
% \wss{Functional, Dynamic, Manual, Automatic, Static etc. Most will
%   be automatic}
					
Initial State: N/A
					
Input: A survey to be filled by the user. Ask them to provide feedback on clarity, 
ease of understanding, and any difficulties encountered.
					
Output: an average score between 1 to 5 for each question.
% \wss{The expected result for the given inputs}

Test Case Derivation: Directly from the requirement.

How test will be performed: After playing around with the visualization, 
users will be asked to fill out the following survey.

\begin{itemize}\renewcommand{\labelitemi}{\scriptsize$\square$}
  \item On a scale of 1 to 5, how clear is the visualization to you?
  \item Are the location of the cart clear to you?
  \item Are the angular position of the pendulum clear to you?
  \item Could you get a sense of acceleration and velocity of the pendulum and cart?
  \item Is the visualization smooth to you?
\end{itemize}

\item{test-vis-3\\}

Type: Automatic
% \wss{Functional, Dynamic, Manual, Automatic, Static etc. Most will
%   be automatic}
					
Initial State: Specified in the \hyperref[table:input_output]{Table \ref*{table:input_output}}.
					
Input: Specified in the \hyperref[table:input_output]{Table \ref*{table:input_output}}.
					
Output: The frame rate of the visualization should be at least 30 frames per second.
% \wss{The expected result for the given inputs}

Test Case Derivation: Directly from the requirement.

How test will be performed: Run the program and measure the frame rate of 
the visualization at every time step.

\end{enumerate}

% \subsubsection{Area of Testing1}

% \wss{It would be nice to have a blurb here to explain why the subsections below
%   cover the requirements.  References to the SRS would be good here.  If a section
%   covers tests for input constraints, you should reference the data constraints
%   table in the SRS.}
		
% \paragraph{Title for Test}

% \begin{enumerate}

% \item{test-id1\\}

% Control: Manual versus Automatic
					
% Initial State: 
					
% Input: 
					
% Output: \wss{The expected result for the given inputs}

% Test Case Derivation: \wss{Justify the expected value given in the Output field}
					
% How test will be performed: 
					
% \item{test-id2\\}

% Control: Manual versus Automatic
					
% Initial State: 
					
% Input: 
					
% Output: \wss{The expected result for the given inputs}

% Test Case Derivation: \wss{Justify the expected value given in the Output field}

% How test will be performed: 

% \end{enumerate}

% \subsubsection{Area of Testing2}

% ...

\subsection{Tests for Nonfunctional Requirements}

\wss{The nonfunctional requirements for accuracy will likely just reference the
  appropriate functional tests from above.  The test cases should mention
  reporting the relative error for these tests.  Not all projects will
  necessarily have nonfunctional requirements related to accuracy}

\wss{Tests related to usability could include conducting a usability test and
  survey.  The survey will be in the Appendix.}

\wss{Static tests, review, inspections, and walkthroughs, will not follow the
format for the tests given below.}

\subsubsection{Accuracy}
		
\paragraph{Accuracy}

\begin{enumerate}

\item{test-acc-1\\}

Type: Automatic

Refer to \hyperref[func_test_simulator]{Section \ref*{func_test_simulator}}.

\end{enumerate}

\subsubsection{Usability}

\paragraph{Usability}

\begin{enumerate}

\item{test-useable-1\\}

Type: Automatic

Refer to \hyperref[func_test_visualization]{Section \ref*{func_test_visualization}}.

\end{enumerate}

\subsubsection{Maintainability}

\paragraph{Maintainability}

\begin{enumerate}

\item{test-maintain-1\\}

Type: Manual
					
Initial State: N/A
					
Input/Condition: N/A
					
Output/Result: A score from experts on the maintainability of the software.
					
How test will be performed: We use survey in the 
\hyperref[sebsec_design_verification_plan]{Section \ref*{sebsec_design_verification_plan}} as 
a measure of maintainability. The design and survey will be given to the experts and 
they will provide feedback on the maintainability of the software.

\end{enumerate}

\subsubsection{Portability}

\paragraph{Portability}

\begin{enumerate}

\item{test-port-1\\}

Type: Manual
					
Initial State: N/A
					
Input/Condition: test-sim-1 and est-vis-2. 
					
Output/Result: Successful test implies portability of the software.
					
How test will be performed: The test will be performed manually 
by executing the software and tests on the Linux and macOS operating systems.

\end{enumerate}

\subsection{Traceability Between Test Cases and Requirements}

\begin{table}[!h]
  \centering
  \caption{Relation of Test Cases and Requirements.}
  \label{tab:traceability}
  \begin{tabular}{|l|l|l|l|l|l|l|l|l|l|}
    \hline
                    & R1 & R2 & R3 & R4 & R5 & NFR1 & NFR2 & NFR3 & NFR4 \\ \hline
    test-inout-1    & X  &    &    &    &    &      &      &      &      \\ \hline
    test-sim-1      &    & X  & X  &    &    &  X   &      &      &      \\ \hline
    test-control-1  &    &    &    & X  &    &      &      &      &      \\ \hline
    test-vis-1      &    &    &    &    & X  &      &      &      &      \\ \hline
    test-vis-2      &    &    &    &    & X  &      &      &      &      \\ \hline
    test-vis-3      &    &    &    &    & X  &      &      &      &      \\ \hline
    test-acc-1      &    &  X &  X &    &    &  X   &      &      &      \\ \hline
    test-useable-1  &    &    &    &    &  X &      &  X   &      &      \\ \hline
    test-maintain-1 &    &    &    &    &    &      &      &  X   &      \\ \hline
    test-port-1     &    &    &    &    &    &      &      &      &  X   \\ \hline
  \end{tabular}
\end{table}

\wss{Provide a table that shows which test cases are supporting which
  requirements.}

\section{Unit Test Description} \label{unit_test}

TBD. This section will be completed when the MIS is finished.

\wss{This section should not be filled in until after the MIS (detailed design
  document) has been completed.}

\wss{Reference your MIS (detailed design document) and explain your overall
philosophy for test case selection.}  

\wss{To save space and time, it may be an option to provide less detail in this section.  
For the unit tests you can potentially layout your testing strategy here.  That is, you 
can explain how tests will be selected for each module.  For instance, your test building 
approach could be test cases for each access program, including one test for normal behaviour 
and as many tests as needed for edge cases.  Rather than create the details of the input 
and output here, you could point to the unit testing code.  For this to work, you code 
needs to be well-documented, with meaningful names for all of the tests.}

\subsection{Unit Testing Scope}

TBD. This section will be completed when the MIS is finished.

\wss{What modules are outside of the scope.  If there are modules that are
  developed by someone else, then you would say here if you aren't planning on
  verifying them.  There may also be modules that are part of your software, but
  have a lower priority for verification than others.  If this is the case,
  explain your rationale for the ranking of module importance.}

\subsection{Tests for Functional Requirements}

TBD. This section will be completed when the MIS is finished.

\wss{Most of the verification will be through automated unit testing.  If
  appropriate specific modules can be verified by a non-testing based
  technique.  That can also be documented in this section.}

% \subsubsection{Module 1}

% \wss{Include a blurb here to explain why the subsections below cover the module.
%   References to the MIS would be good.  You will want tests from a black box
%   perspective and from a white box perspective.  Explain to the reader how the
%   tests were selected.}

% \begin{enumerate}

% \item{test-id1\\}

% Type: \wss{Functional, Dynamic, Manual, Automatic, Static etc. Most will
%   be automatic}
					
% Initial State: 
					
% Input: 
					
% Output: \wss{The expected result for the given inputs}

% Test Case Derivation: \wss{Justify the expected value given in the Output field}

% How test will be performed: 
					
% \item{test-id2\\}

% Type: \wss{Functional, Dynamic, Manual, Automatic, Static etc. Most will
%   be automatic}
					
% Initial State: 
					
% Input: 
					
% Output: \wss{The expected result for the given inputs}

% Test Case Derivation: \wss{Justify the expected value given in the Output field}

% How test will be performed: 

% \item{...\\}
    
% \end{enumerate}

% \subsubsection{Module 2}

% ...

\subsection{Tests for Nonfunctional Requirements}

TBD. This section will be completed when the MIS is finished.

\wss{If there is a module that needs to be independently assessed for
  performance, those test cases can go here.  In some projects, planning for
  nonfunctional tests of units will not be that relevant.}

\wss{These tests may involve collecting performance data from previously
  mentioned functional tests.}

% \subsubsection{Module ?}
		
% \begin{enumerate}

% \item{test-id1\\}

% Type: \wss{Functional, Dynamic, Manual, Automatic, Static etc. Most will
%   be automatic}
					
% Initial State: 
					
% Input/Condition: 
					
% Output/Result: 
					
% How test will be performed: 
					
% \item{test-id2\\}

% Type: Functional, Dynamic, Manual, Static etc.
					
% Initial State: 
					
% Input: 
					
% Output: 
					
% How test will be performed: 

% \end{enumerate}

% \subsubsection{Module ?}

% ...

\subsection{Traceability Between Test Cases and Modules}

TBD. This section will be completed when the MIS is finished.

\wss{Provide evidence that all of the modules have been considered.}

\newpage
\bibliographystyle{plainnat}

\bibliography{../../refs/References}

\newpage

% \section{Appendix}

% This is where you can place additional information.

% \subsection{Symbolic Parameters}

% The definition of the test cases will call for SYMBOLIC\_CONSTANTS.
% Their values are defined in this section for easy maintenance.

% \subsection{Usability Survey Questions?}

% \wss{This is a section that would be appropriate for some projects.}

% \newpage{}
% \section*{Appendix --- Reflection}

% The information in this section will be used to evaluate the team members on the
% graduate attribute of Lifelong Learning.  Please answer the following questions:

% \newpage{}
% \section*{Appendix --- Reflection}

% \wss{This section is not required for CAS 741}

% The information in this section will be used to evaluate the team members on the
% graduate attribute of Lifelong Learning.  Please answer the following questions:

% \begin{enumerate}
%   \item What knowledge and skills will the team collectively need to acquire to
%   successfully complete the verification and validation of your project?
%   Examples of possible knowledge and skills include dynamic testing knowledge,
%   static testing knowledge, specific tool usage etc.  You should look to
%   identify at least one item for each team member.
%   \item For each of the knowledge areas and skills identified in the previous
%   question, what are at least two approaches to acquiring the knowledge or
%   mastering the skill?  Of the identified approaches, which will each team
%   member pursue, and why did they make this choice?
% \end{enumerate}

\end{document}