\documentclass{article}

\usepackage{tabularx}
\usepackage{booktabs}
\usepackage{amsmath}

\title{Problem Statement and Goals\\\progname}

\author{\authname}

\date{Jan 19, 2024}

%% Comments

\usepackage{color}

%\newif\ifcomments\commentstrue %displays comments
\newif\ifcomments\commentsfalse %so that comments do not display

\ifcomments
\newcommand{\authornote}[3]{\textcolor{#1}{[#3 ---#2]}}
\newcommand{\todo}[1]{\textcolor{red}{[TODO: #1]}}
\else
\newcommand{\authornote}[3]{}
\newcommand{\todo}[1]{}
\fi

\newcommand{\wss}[1]{\authornote{blue}{SS}{#1}} 
\newcommand{\plt}[1]{\authornote{magenta}{TPLT}{#1}} %For explanation of the template
\newcommand{\an}[1]{\authornote{cyan}{Author}{#1}}

%% Common Parts

\newcommand{\progname}{Inverted Pendulum Control Systems} % PUT YOUR PROGRAM NAME HERE
\newcommand{\authname}{Morteza Mirzaei} % AUTHOR NAMES                  

\usepackage{hyperref}
    \hypersetup{colorlinks=true, linkcolor=blue, citecolor=blue, filecolor=blue,
                urlcolor=blue, unicode=false}
    \urlstyle{same}
                                


\begin{document}

\maketitle

\begin{table}[hp]
\caption{Revision History} \label{TblRevisionHistory}
\begin{tabularx}{\textwidth}{llX}
\toprule
\textbf{Date} & \textbf{Developer(s)} & \textbf{Change}\\
\midrule
2024-01-19 & Morteza Mirzaei & Write the first version of the initial document for reverse pendulum.\\
\bottomrule
\end{tabularx}
\end{table}

\section{Problem Statement}

The inverted pendulum problem is a classic challenge in control theory,
involving the stabilization of an upright pendulum atop a moving pivot.
Its inherent instability makes it a compelling theoretical exercise,
demanding precise control to prevent the pendulum from falling.
This problem acts as a benchmark for testing and refining control algorithms,
influencing advancements in areas like autonomous vehicles, robotics,
and stability control for devices such as self-balancing scooters.
The insights gained from solving this problem have practical applications,
shaping developments in real-world systems that demand stability
and dynamic control.

% \wss{You should check your problem statement with the
% \href{https://github.com/smiths/capTemplate/blob/main/docs/Checklists/ProbState-Checklist.pdf}
% {problem statement checklist}.}
% \wss{You can change the section headings, as long as you include the required information.}

\subsection{Problem}
Mathematically, the dynamics of an inverted pendulum can be described
by a set of differential equations derived from Newton's laws of motion.
Let \(x\) be the horizontal position of the cart, \(\theta\) the angle
of the pendulum with respect to the vertical axis.
The equations of motion for the inverted pendulum are given by:
\[
(M+m)\ddot{x} -ml \ddot{\theta} \cos \theta + ml \dot{\theta}^2 \sin \theta = F
\]
Where: \(M\) is the mass of the cart,
\(m\) is the mass of the pendulum point mass,
\(g\) is the acceleration due to gravity,
\(F\) is the horizontal force applied to the cart,
and \(L\) is the length of the pendulum.

The challenge in the inverted pendulum problem is to design
a control law \(F\) such that the system remains stable,
and the pendulum stays balanced in the inverted position
despite its inherent instability.


\subsection{Inputs and Outputs}

The inverted pendulum problem involves the following key inputs:
\begin{enumerate}
    \item Angle of the pendulum with the axis perpendicular to the cart (\(\theta\))
    \item Angular velocity of the pendulum (\(\dot{\theta}\))
    \item Position of the cart along the axis (\(x\))
    \item Speed of the cart along the axis (\(\dot{x}\))
\end{enumerate}

The primary output of the inverted pendulum problem is the force (\(F\)),
representing the force applied to the cart along the \(x\)-axis.


% \wss{Characterize the problem in terms of ``high level'' inputs and outputs.  
% Use abstraction so that you can avoid details.}

% \subsection{Stakeholders}

% \subsection{Environment}

% \wss{Hardware and software}

\section{Goals}

\begin{enumerate}
    \item \textbf{Stability Achievement:}
    Develop a control algorithm that effectively stabilizes
    the inverted pendulum system, maintaining the pendulum
    within a deviation of less than 20 degrees from the upright position
    after 20 seconds of operation.

    \item \textbf{Real-time Responsiveness:}
    Design the controller to exhibit real-time responsiveness,
    ensuring swift and accurate adjustments to the applied forces
    based on feedback from sensors, with the requirement that the angular speed
    remains below 10 degrees per second for smooth and controlled movements.

    \item \textbf{Recovery from Instability:}
    Implement a feature that enables the system to autonomously recover
    from instability caused by external forces, bringing the pendulum back
    to a stable position within 20 seconds of deviation.
\end{enumerate}

\section{Stretch Goals}
\begin{enumerate}
    \item \textbf{Generalizability:}
    Extend the control solution's versatility beyond the traditional
    simple round-shaped pendulum to encompass more advanced shapes,
    specifically focusing on configurations where the center of
    mass of the complex pendulum dynamically changes through rotation.
    This goal aims to enhance the control algorithm's adaptability,
    allowing it to effectively handle a broader range of inverted
    pendulum systems with varying complexities and geometries.
\end{enumerate}

\end{document}