\documentclass[12pt, titlepage]{article}

\usepackage{amsmath, mathtools}

\usepackage[round]{natbib}
\usepackage{amsfonts}
\usepackage{amssymb}
\usepackage{graphicx}
\usepackage{colortbl}
\usepackage{xr}
\usepackage{hyperref}
\usepackage{longtable}
\usepackage{xfrac}
\usepackage{tabularx}
\usepackage{float}
\usepackage{siunitx}
\usepackage{booktabs}
\usepackage{multirow}
\usepackage[section]{placeins}
\usepackage{caption}
\usepackage{fullpage}

\hypersetup{
bookmarks=true,     % show bookmarks bar?
colorlinks=true,       % false: boxed links; true: colored links
linkcolor=red,          % color of internal links (change box color with linkbordercolor)
citecolor=blue,      % color of links to bibliography
filecolor=magenta,  % color of file links
urlcolor=cyan          % color of external links
}

\usepackage{array}

\externaldocument{../../SRS/SRS}

%% Comments

\usepackage{color}

%\newif\ifcomments\commentstrue %displays comments
\newif\ifcomments\commentsfalse %so that comments do not display

\ifcomments
\newcommand{\authornote}[3]{\textcolor{#1}{[#3 ---#2]}}
\newcommand{\todo}[1]{\textcolor{red}{[TODO: #1]}}
\else
\newcommand{\authornote}[3]{}
\newcommand{\todo}[1]{}
\fi

\newcommand{\wss}[1]{\authornote{blue}{SS}{#1}} 
\newcommand{\plt}[1]{\authornote{magenta}{TPLT}{#1}} %For explanation of the template
\newcommand{\an}[1]{\authornote{cyan}{Author}{#1}}

%% Common Parts

\newcommand{\progname}{Inverted Pendulum Control Systems} % PUT YOUR PROGRAM NAME HERE
\newcommand{\authname}{Morteza Mirzaei} % AUTHOR NAMES                  

\usepackage{hyperref}
    \hypersetup{colorlinks=true, linkcolor=blue, citecolor=blue, filecolor=blue,
                urlcolor=blue, unicode=false}
    \urlstyle{same}
                                


\begin{document}

\title{Module Interface Specification for \progname{}}

\author{\authname}

\date{\today}

\maketitle

\pagenumbering{roman}

\section{Revision History}

\begin{tabularx}{\textwidth}{p{3cm}p{2cm}X}
\toprule {\bf Date} & {\bf Version} & {\bf Notes}\\
\midrule
2024-03-18 & 1.0 & Initial Release.\\
\bottomrule
\end{tabularx}

~\newpage

\section{Symbols, Abbreviations and Acronyms}

See SRS Documentation at: \\
\url{https://github.com/mirzaim/ipcs/blob/main/docs/SRS/SRS.pdf}

\wss{Also add any additional symbols, abbreviations or acronyms}

\newpage

\tableofcontents

\newpage

\pagenumbering{arabic}

\section{Introduction}

The following document details the Module Interface Specifications for
\progname{}.

Complementary documents include the System Requirement Specifications
and Module Guide.  The full documentation and implementation can be
found at \url{https://github.com/mirzaim/ipcs}. 
\wss{provide the url for your repo}

\section{Notation}

\wss{You should describe your notation.  You can use what is below as
  a starting point.}

The structure of the MIS for modules comes from \citet{HoffmanAndStrooper1995},
with the addition that template modules have been adapted from
\cite{GhezziEtAl2003}.  The mathematical notation comes from Chapter 3 of
\citet{HoffmanAndStrooper1995}.  For instance, the symbol := is used for a
multiple assignment statement and conditional rules follow the form $(c_1
\Rightarrow r_1 | c_2 \Rightarrow r_2 | ... | c_n \Rightarrow r_n )$.

The following table summarizes the primitive data types used by \progname. 

\begin{center}
\renewcommand{\arraystretch}{1.2}
\noindent 
\begin{tabular}{l l p{7.5cm}} 
\toprule 
\textbf{Data Type} & \textbf{Notation} & \textbf{Description}\\ 
\midrule
character & char & a single symbol or digit\\
string & $\text{char}^*$ & a sequence of characters\\
boolean & $\mathbb{B}$ & a binary variable that can be either true or false\\
integer & $\mathbb{Z}$ & a number without a fractional component in (-$\infty$, $\infty$) \\
natural number & $\mathbb{N}$ & a number without a fractional component in [1, $\infty$) \\
real & $\mathbb{R}$ & any number in (-$\infty$, $\infty$)\\
linguistic variable & ling\_var & a linguistic variable that used in the fuzzy logic rules. 
                               Its type is string.\\
rule & rule & a rule that used in the fuzzy logic. Its type is string.\\
fuzzy set & fuzzy\_set & a fuzzy set that is determined by the membership function ($R \rightarrow R$).\\
array & Array[T] & a sequence of elements of type T.\\
dictionary & Dict[K, V] & a collection of key-value pairs, where K is the key type and V is the value type.\\
\bottomrule
\end{tabular} 
\end{center}

\noindent
The specification of \progname \ uses some derived data types: sequences, strings, and
tuples. Sequences are lists filled with elements of the same data type. Strings
are sequences of characters. Tuples contain a list of values, potentially of
different types. In addition, \progname \ uses functions, which
are defined by the data types of their inputs and outputs. Local functions are
described by giving their type signature followed by their specification.

\section{Module Decomposition}

The following table is taken directly from the Module Guide document for this project.

\begin{table}[h!]
\centering
\begin{tabular}{p{0.3\textwidth} p{0.6\textwidth}}
\toprule
\textbf{Level 1} & \textbf{Level 2}\\
\midrule

{Hardware-Hiding Module} & ~ \\
\midrule

\multirow{7}{0.3\textwidth}{Behaviour-Hiding Module}
& Fuzzy Logic Module\\
& Config Reader Module\\
% & Input Parameter Module\\
% & Motion ODEs Module\\ 
& World Checker Module\\
& Simulator Module\\
& World Module\\
& Manager Module\\
& Main Module\\
\midrule

\multirow{3}{0.3\textwidth}{Software Decision Module}
& Controller Module\\
& GUI Module\\
& File Output Module\\
% & ODE Solver Module\\
% & Sequence Data Structure Module\\
\bottomrule

\end{tabular}
\caption{Module Hierarchy}
\label{TblMH}
\end{table}

\newpage
~\newpage

% \section{MIS of \wss{Module Name}} \label{Module} \wss{Use labels for
%   cross-referencing}

% \wss{You can reference SRS labels, such as R\ref{R_Inputs}.}

% \wss{It is also possible to use \LaTeX for hypperlinks to external documents.}

% \subsection{Module}

% \wss{Short name for the module}

% \subsection{Uses}


% \subsection{Syntax}

% \subsubsection{Exported Constants}

% \subsubsection{Exported Access Programs}

% \begin{center}
% \begin{tabular}{p{2cm} p{4cm} p{4cm} p{2cm}}
% \hline
% \textbf{Name} & \textbf{In} & \textbf{Out} & \textbf{Exceptions} \\
% \hline
% \wss{accessProg} & - & - & - \\
% \hline
% \end{tabular}
% \end{center}

% \subsection{Semantics}

% \subsubsection{State Variables}

% \wss{Not all modules will have state variables.  State variables give the module
%   a memory.}

% \subsubsection{Environment Variables}

% \wss{This section is not necessary for all modules.  Its purpose is to capture
%   when the module has external interaction with the environment, such as for a
%   device driver, screen interface, keyboard, file, etc.}

% \subsubsection{Assumptions}

% \wss{Try to minimize assumptions and anticipate programmer errors via
%   exceptions, but for practical purposes assumptions are sometimes appropriate.}

% \subsubsection{Access Routine Semantics}

% \noindent \wss{accessProg}():
% \begin{itemize}
% \item transition: \wss{if appropriate} 
% \item output: \wss{if appropriate} 
% \item exception: \wss{if appropriate} 
% \end{itemize}

% \wss{A module without environment variables or state variables is unlikely to
%   have a state transition.  In this case a state transition can only occur if
%   the module is changing the state of another module.}

% \wss{Modules rarely have both a transition and an output.  In most cases you
%   will have one or the other.}

% \subsubsection{Local Functions}

% \wss{As appropriate} \wss{These functions are for the purpose of specification.
%   They are not necessarily something that is going to be implemented
%   explicitly.  Even if they are implemented, they are not exported; they only
%   have local scope.}

% \newpage

\section{MIS of Fuzzy Logic Module} \label{mFuzzyLogic} \wss{Use labels for
  cross-referencing}

\wss{You can reference SRS labels, such as R\ref{R_Inputs}.}

\wss{It is also possible to use \LaTeX for hypperlinks to external documents.}

\subsection{Fuzzy Logic}
In this section, we provide some background information on fuzzy logic and 
fuzzy controllers that will be used in the \progname{} project.

\subsubsection{Fuzzy Sets}
In the realm of control systems, traditional methods often struggle 
to handle the inherent uncertainty and imprecision present in real-world data. 
Fuzzy logic offers a solution by introducing the concept of fuzzy sets. 
Unlike crisp sets, where elements are either fully inside or outside a set, 
fuzzy sets allow for partial membership. This means that an element can 
belong to a set to a certain degree, reflecting the uncertainty or 
vagueness inherent in many real-world scenarios. A fuzzycontroller
is a control system based on fuzzy logic, which uses fuzzy sets to
represent linguistic variables and rules to make decisions.
It consits of three main steps: fuzzification, rule processing, and
defuzzification that are described below.

\subsubsection{Fuzzy Controller Steps}

\begin{enumerate}
  \item \textbf{Fuzzification:}
  Fuzzification is the initial step in a fuzzy controller where
  crisp inputs are transformed into fuzzy sets. This process 
  involves mapping numerical inputs to linguistic variables and 
  assigning membership degrees to each linguistic variable's 
  corresponding fuzzy set. For instance, in a temperature 
  control system, a crisp input value like "25 degrees Celsius" 
  might be fuzzified into linguistic variables such as "cold," 
  "warm," and "hot," with each linguistic variable having 
  membership degrees reflecting the degree to which the 
  input belongs to each fuzzy set.
   
  \item \textbf{Rule Processing:}
  Once the inputs are fuzzified, fuzzy rules are applied 
  to determine the appropriate control actions. These rules, 
  often expressed in the form of IF-THEN statements, map combinations 
  of fuzzy input variables to fuzzy output variables. For example, 
  a rule might state "IF temperature is cold AND humidity is high 
  THEN heater is high." By evaluating these rules and considering 
  the fuzzy input values, the fuzzy controller determines the 
  appropriate control action to take.
   
  \item \textbf{Defuzzification:}
  After applying fuzzy rules and obtaining fuzzy output sets, 
  the final step is defuzzification. Here, the fuzzy output sets 
  are aggregated to produce a crisp output value that represents 
  the control action to be taken. Various defuzzification 
  methods exist, such as centroid defuzzification or maximum 
  membership principle. These methods translate the fuzzy output 
  sets back into a single numerical value that guides the system's response.
\end{enumerate}

In conclusion, the fuzzy controller method offers a flexible and effective approach to control systems, allowing for the incorporation of human-like reasoning and handling uncertainty inherent in real-world environments.

\subsection{Module}

Fuzzy Logic
\wss{Short name for the module}

\subsection{Uses}

parser module (Section \ref{mParser})

\subsection{Syntax}

% \subsubsection{Exported Constants}

\subsubsection{Exported Access Programs}

\begin{center}
\begin{tabular}{p{3cm} p{5cm} p{2cm} p{2cm}}
\hline
\textbf{Name} & \textbf{In} & \textbf{Out} & \textbf{Exceptions} \\
\hline
decide & World & $\mathbb{R}$ & - \\
load\_fuzzy\_rules & fuzzy\_sets: 
\textit{Array}[$\text{ling\_var} \times \text{fuzzy\_set}$], 
rules: \textit{Array}[\text{rule}] & - & - \\
\hline
\end{tabular}
\end{center}

\subsection{Semantics}

\subsubsection{State Variables}

fuzzyInputSets: \textit{Array}[$\text{ling\_var} \times \text{fuzzy\_set}$] \\
fuzzyOutputSets: \textit{Array}[$\text{ling\_var} \times \text{fuzzy\_set}$] \\
fuzzyRules: \textit{Array}[\text{rule}] \\

\wss{Not all modules will have state variables.  State variables give the module
  a memory.}

\subsubsection{Environment Variables}

None
\wss{This section is not necessary for all modules.  Its purpose is to capture
  when the module has external interaction with the environment, such as for a
  device driver, screen interface, keyboard, file, etc.}

\subsubsection{Assumptions}

None
\wss{Try to minimize assumptions and anticipate programmer errors via
  exceptions, but for practical purposes assumptions are sometimes appropriate.}

\subsubsection{Access Routine Semantics}

\noindent decide(World):
\begin{itemize}
\item transition: -
\item output: returns the force value.
      \begin{enumerate}
        \item In the first step, it fuzzifies the world state $x$, $\theta$, $\dot{x}$,
        and $\dot{\theta}$ to get the fuzzy input sets. This step uses fuzzyInputSets
        environment variable and measures the membership degree of each linguistic
        variable's corresponding fuzzy set.
        \item In the second step, it applies the fuzzy rules on fuzzified input sets to
        get the fuzzy output sets. This step uses fuzzyRules environment variable.
        At the end of this step we have membership degrees of each linguistic 
        variable's corresponding fuzzyOutputSets.
        \item In the third step, it defuzzifies the fuzzy output sets to get the force
        value. In this step we use center of mass method as defuzzification method.
      \end{enumerate}
\item exception: -
\end{itemize}

\noindent load\_fuzzy\_rules(fuzzy\_sets, rules):
\begin{itemize}
\item transition: 
      It sets the fuzzyInputSets, fuzzyOutputSets and fuzzyRules state variables.
\item output: -
\item exception: -
\end{itemize}

\wss{A module without environment variables or state variables is unlikely to
  have a state transition.  In this case a state transition can only occur if
  the module is changing the state of another module.}

\wss{Modules rarely have both a transition and an output.  In most cases you
  will have one or the other.}

% \subsubsection{Local Functions}

% \wss{As appropriate} \wss{These functions are for the purpose of specification.
%   They are not necessarily something that is going to be implemented
%   explicitly.  Even if they are implemented, they are not exported; they only
%   have local scope.}

\newpage

\section{MIS of Parser Module} \label{mParser} \wss{Use labels for
  cross-referencing}

\wss{You can reference SRS labels, such as R\ref{R_Inputs}.}

\wss{It is also possible to use \LaTeX for hypperlinks to external documents.}

\subsection{Module}

Parser

\wss{Short name for the module}

\subsection{Uses}
None

\subsection{Syntax}

% \subsubsection{Exported Constants}

\subsubsection{Exported Access Programs}

\begin{center}
\begin{tabular}{p{3cm} p{2cm} p{5cm} p{2cm}}
\hline
\textbf{Name} & \textbf{In} & \textbf{Out} & \textbf{Exceptions} \\
\hline
load\_config & file\_path: $\text{char}^*$ & World & FileNotFoundException, InvalidFileFormatException \\
load\_Fuzzy\_logic & file\_path: $\text{chat}^*$ & 
fuzzy\_sets: \textit{Array}[$\text{ling\_var} \times \text{fuzzy\_set}$],
rules: \textit{Array}[\text{rule}] & FileNotFoundException, InvalidFileFormatException \\
\hline
\end{tabular}
\end{center}

\subsection{Semantics}

\subsubsection{State Variables}

None
\wss{Not all modules will have state variables.  State variables give the module
  a memory.}

\subsubsection{Environment Variables}

file: A text file.
\wss{This section is not necessary for all modules.  Its purpose is to capture
  when the module has external interaction with the environment, such as for a
  device driver, screen interface, keyboard, file, etc.}

\subsubsection{Assumptions}

None
\wss{Try to minimize assumptions and anticipate programmer errors via
  exceptions, but for practical purposes assumptions are sometimes appropriate.}

\subsubsection{Access Routine Semantics}

\noindent load\_config(file\_path):
\begin{itemize}
\item transition: -
\item output: Read initial world configuration from the file and returns a
      World.
\item exception:
      $exc := ( \\
      \text{if } \text{file\_path} \text{ does not exist then }
      \text{FileNotFoundException} \\
      | \\
      \text{if } \text{file\_path} \text{ is not in the correct format then }
      \text{InvalidFileFormatException} \\
      )$
\end{itemize}

\noindent load\_Fuzzy\_logic(file\_path):
\begin{itemize}
\item transition: -
\item output: It acts like a complier and reads the file and returns the 
      fuzzy sets and rules. returns: \\
      fuzzy\_sets: Array[LinguisticVariable $\times$ FuzzySet], rules: fuzzyRules: Array[Rule]
\item exception: 
      $exc := ( \\
      \text{if } \text{file\_path} \text{ does not exist then }
      \text{FileNotFoundException} \\
      | \\
      \text{if } \text{file\_path} \text{ is not in the correct format then }
      \text{InvalidFileFormatException} \\
      )$
\end{itemize}

\wss{A module without environment variables or state variables is unlikely to
  have a state transition.  In this case a state transition can only occur if
  the module is changing the state of another module.}

\wss{Modules rarely have both a transition and an output.  In most cases you
  will have one or the other.}

% \subsubsection{Local Functions}

% \wss{As appropriate} \wss{These functions are for the purpose of specification.
%   They are not necessarily something that is going to be implemented
%   explicitly.  Even if they are implemented, they are not exported; they only
%   have local scope.}

\newpage

% \section{MIS of Input Parameter Module} \label{mInputParamter} \wss{Use labels for
%   cross-referencing}

% \wss{You can reference SRS labels, such as R\ref{R_Inputs}.}

% \wss{It is also possible to use \LaTeX for hypperlinks to external documents.}

% \subsection{Module}

% \wss{Short name for the module}

% \subsection{Uses}


% \subsection{Syntax}

% \subsubsection{Exported Constants}

% \subsubsection{Exported Access Programs}

% \begin{center}
% \begin{tabular}{p{2cm} p{4cm} p{4cm} p{2cm}}
% \hline
% \textbf{Name} & \textbf{In} & \textbf{Out} & \textbf{Exceptions} \\
% \hline
% \wss{accessProg} & - & - & - \\
% \hline
% \end{tabular}
% \end{center}

% \subsection{Semantics}

% \subsubsection{State Variables}

% \wss{Not all modules will have state variables.  State variables give the module
%   a memory.}

% \subsubsection{Environment Variables}

% \wss{This section is not necessary for all modules.  Its purpose is to capture
%   when the module has external interaction with the environment, such as for a
%   device driver, screen interface, keyboard, file, etc.}

% \subsubsection{Assumptions}

% \wss{Try to minimize assumptions and anticipate programmer errors via
%   exceptions, but for practical purposes assumptions are sometimes appropriate.}

% \subsubsection{Access Routine Semantics}

% \noindent \wss{accessProg}():
% \begin{itemize}
% \item transition: \wss{if appropriate} 
% \item output: \wss{if appropriate} 
% \item exception: \wss{if appropriate} 
% \end{itemize}

% \wss{A module without environment variables or state variables is unlikely to
%   have a state transition.  In this case a state transition can only occur if
%   the module is changing the state of another module.}

% \wss{Modules rarely have both a transition and an output.  In most cases you
%   will have one or the other.}

% \subsubsection{Local Functions}

% \wss{As appropriate} \wss{These functions are for the purpose of specification.
%   They are not necessarily something that is going to be implemented
%   explicitly.  Even if they are implemented, they are not exported; they only
%   have local scope.}

% \newpage

% \section{MIS of Motion ODEs} \label{mMotionODEs} \wss{Use labels for
%   cross-referencing}

% \wss{You can reference SRS labels, such as R\ref{R_Inputs}.}

% \wss{It is also possible to use \LaTeX for hypperlinks to external documents.}

% \subsection{Module}

% Motion ODEs

% \wss{Short name for the module}

% \subsection{Uses}
% None

% \subsection{Syntax}

% % \subsubsection{Exported Constants}

% \subsubsection{Exported Access Programs}

% \begin{center}
% \begin{tabular}{p{2cm} p{4cm} p{4cm} p{2cm}}
% \hline
% \textbf{Name} & \textbf{In} & \textbf{Out} & \textbf{Exceptions} \\
% \hline
% ODE\_Motion & - & $(\mathbb{R} \rightarrow \mathbb{R}) \rightarrow (\mathbb{R} \rightarrow \mathbb{R})^2$ & - \\
% ODE\_Motion & - & $(\mathbb{R}^3 \rightarrow \mathbb{R})^2$ & - \\
% \hline
% \end{tabular}
% \end{center}

% \subsection{Semantics}

% \subsubsection{State Variables}

% \wss{Not all modules will have state variables.  State variables give the module
%   a memory.}

% \subsubsection{Environment Variables}

% \wss{This section is not necessary for all modules.  Its purpose is to capture
%   when the module has external interaction with the environment, such as for a
%   device driver, screen interface, keyboard, file, etc.}

% \subsubsection{Assumptions}

% \wss{Try to minimize assumptions and anticipate programmer errors via
%   exceptions, but for practical purposes assumptions are sometimes appropriate.}

% \subsubsection{Access Routine Semantics}

% \noindent \wss{accessProg}():
% \begin{itemize}
% \item transition: \wss{if appropriate} 
% \item output: $x(t)$ and $\theta(t)$, such that 
%       $F(t)$ = $(M+m)\ddot{x}(t) - ml\ddot{\theta}(t)\cos{\theta(t)} + 
%       ml (\dot{\theta}^2(t)) \sin{\theta(t)}$ and $l \ddot{\theta}(t) + 
%       g \sin{\theta(t)} = \ddot{x}(t) \cos{\theta(t)}$ holds.
% \item exception: \wss{if appropriate} 
% \end{itemize}

% \wss{A module without environment variables or state variables is unlikely to
%   have a state transition.  In this case a state transition can only occur if
%   the module is changing the state of another module.}

% \wss{Modules rarely have both a transition and an output.  In most cases you
%   will have one or the other.}

% \subsubsection{Local Functions}

% \wss{As appropriate} \wss{These functions are for the purpose of specification.
%   They are not necessarily something that is going to be implemented
%   explicitly.  Even if they are implemented, they are not exported; they only
%   have local scope.}

% \newpage

\section{MIS of Output Verification Module} \label{mOutputVerification} \wss{Use labels for
  cross-referencing}

\wss{You can reference SRS labels, such as R\ref{R_Inputs}.}

\wss{It is also possible to use \LaTeX for hypperlinks to external documents.}

\subsection{Module}

Output Verification

\wss{Short name for the module}

\subsection{Uses}

World Module (Section \ref{mWorld})

\subsection{Syntax}

% \subsubsection{Exported Constants}

\subsubsection{Exported Access Programs}

\begin{center}
\begin{tabular}{p{2cm} p{4cm} p{4cm} p{2cm}}
\hline
\textbf{Name} & \textbf{In} & \textbf{Out} & \textbf{Exceptions} \\
\hline
verify & world & - & InvalidAngleException, InvalidPositionException \\
\hline
\end{tabular}
\end{center}

\subsection{Semantics}

\subsubsection{State Variables}

None

\wss{Not all modules will have state variables.  State variables give the module
  a memory.}

\subsubsection{Environment Variables}

None

\wss{This section is not necessary for all modules.  Its purpose is to capture
  when the module has external interaction with the environment, such as for a
  device driver, screen interface, keyboard, file, etc.}

\subsubsection{Assumptions}

None

\wss{Try to minimize assumptions and anticipate programmer errors via
  exceptions, but for practical purposes assumptions are sometimes appropriate.}

\subsubsection{Access Routine Semantics}

\noindent verify(world):
\begin{itemize}
\item transition: -
\item output: -
\item exception: 
      $exc := ( \\
      \text{if } \theta \leq -\pi \text{ or } \theta \geq \pi \text{ then }
      \text{InvalidAngleException} \\
      | \\
      \text{if } x \leq \text{min\_x} \text{ or } x \geq \text{max\_x} 
      \text{ then } \text{InvalidPositionException}\\
      )$
\end{itemize}

\wss{A module without environment variables or state variables is unlikely to
  have a state transition.  In this case a state transition can only occur if
  the module is changing the state of another module.}

\wss{Modules rarely have both a transition and an output.  In most cases you
  will have one or the other.}

% \subsubsection{Local Functions}

% \wss{As appropriate} \wss{These functions are for the purpose of specification.
%   They are not necessarily something that is going to be implemented
%   explicitly.  Even if they are implemented, they are not exported; they only
%   have local scope.}

\newpage

\section{MIS of Simulator Module} \label{mSimulator} \wss{Use labels for
  cross-referencing}

\wss{You can reference SRS labels, such as R\ref{R_Inputs}.}

\wss{It is also possible to use \LaTeX for hypperlinks to external documents.}

\subsection{Module}

\wss{Short name for the module}
Simulator

\subsection{Uses}

World Module (Section \ref{mWorld}) \\

\subsection{Syntax}

% \subsubsection{Exported Constants}

\subsubsection{Exported Access Programs}

\begin{center}
\begin{tabular}{p{2cm} p{4cm} p{4cm} p{2cm}}
\hline
\textbf{Name} & \textbf{In} & \textbf{Out} & \textbf{Exceptions} \\
\hline
% \wss{accessProg} & - & - & - \\
tick & R & - & - \\
\hline
\end{tabular}
\end{center}

\subsection{Semantics}

\subsubsection{State Variables}

world: World

% \wss{Not all modules will have state variables.  State variables give the module
%   a memory.}

\subsubsection{Environment Variables}

None
% \wss{This section is not necessary for all modules.  Its purpose is to capture
%   when the module has external interaction with the environment, such as for a
%   device driver, screen interface, keyboard, file, etc.}

\subsubsection{Assumptions}

None
% \wss{Try to minimize assumptions and anticipate programmer errors via
%   exceptions, but for practical purposes assumptions are sometimes appropriate.}

\subsubsection{Access Routine Semantics}

\noindent tick(dt):
\begin{itemize}
\item transition:
      It updates the world state by one time step of size $dt$.
      It follows the following steps:
      \begin{itemize}
        \item Caclulate and update the acceleration of the cart and the pendulum.
        \item Update teh velocity of the cart and the pendulum.
        \item Update the position of the cart and the pendulum.
        \item Reset the forces acting on the cart and the pendulum.
      \end{itemize}
      It uses the two following equations to update the world state:
      \begin{gather*}
        F(t) = (M+m)\ddot{x}(t) - ml\ddot{\theta}(t)\cos{\theta(t)} 
        + ml (\dot{\theta}^2(t)) \sin{\theta(t)} \\
        l \ddot{\theta}(t) + g \sin{\theta(t)} = \ddot{x}(t) \cos{\theta(t)}
      \end{gather*}
\item output: -
\item exception: -
\end{itemize}

\wss{A module without environment variables or state variables is unlikely to
  have a state transition.  In this case a state transition can only occur if
  the module is changing the state of another module.}

\wss{Modules rarely have both a transition and an output.  In most cases you
  will have one or the other.}

% \subsubsection{Local Functions}

% \wss{As appropriate} \wss{These functions are for the purpose of specification.
%   They are not necessarily something that is going to be implemented
%   explicitly.  Even if they are implemented, they are not exported; they only
%   have local scope.}

\newpage

\section{MIS of World Module} \label{mWorld} \wss{Use labels for
  cross-referencing}

\wss{You can reference SRS labels, such as R\ref{R_Inputs}.}

\wss{It is also possible to use \LaTeX for hypperlinks to external documents.}

\subsection{Module}

\wss{Short name for the module}
World

\subsection{Uses}

Parser Module (Section \ref{mParser})

\subsection{Syntax}

% \subsubsection{Exported Constants}

\subsubsection{Exported Access Programs}



\begin{center}
\begin{tabular}{p{2cm} p{4cm} p{4cm} p{2cm}}
\hline
\textbf{Name} & \textbf{In} & \textbf{Out} & \textbf{Exceptions} \\
\hline
- & - & - & - \\
\hline
\end{tabular}
\end{center}

\subsection{Semantics}

\subsubsection{State Variables}

m: $\mathbb{R}$ \\
M: $\mathbb{R}$ \\
l: $\mathbb{R}$ \\
g: $\mathbb{R}$ \\
x: $\mathbb{R}$ \\
$\dot{x}$: $\mathbb{R}$ \\
$\ddot{x}$: $\mathbb{R}$ \\
$\theta$: $\mathbb{R}$ \\
$\dot{\theta}$: $\mathbb{R}$ \\
$\ddot{\theta}$: $\mathbb{R}$ \\
F: $\mathbb{R}$ \\

\wss{Not all modules will have state variables.  State variables give the module
  a memory.}

\subsubsection{Environment Variables}

None

\wss{This section is not necessary for all modules.  Its purpose is to capture
  when the module has external interaction with the environment, such as for a
  device driver, screen interface, keyboard, file, etc.}

\subsubsection{Assumptions}

None

\wss{Try to minimize assumptions and anticipate programmer errors via
  exceptions, but for practical purposes assumptions are sometimes appropriate.}

\subsubsection{Access Routine Semantics}

None
% \noindent \wss{accessProg}():
% \begin{itemize}
% \item transition: \wss{if appropriate} 
% \item output: \wss{if appropriate} 
% \item exception: \wss{if appropriate} 
% \end{itemize}

\wss{A module without environment variables or state variables is unlikely to
  have a state transition.  In this case a state transition can only occur if
  the module is changing the state of another module.}

\wss{Modules rarely have both a transition and an output.  In most cases you
  will have one or the other.}

% \subsubsection{Local Functions}

% \wss{As appropriate} \wss{These functions are for the purpose of specification.
%   They are not necessarily something that is going to be implemented
%   explicitly.  Even if they are implemented, they are not exported; they only
%   have local scope.}

\newpage

\section{MIS of Controller Module} \label{mController} \wss{Use labels for
  cross-referencing}

\wss{You can reference SRS labels, such as R\ref{R_Inputs}.}

\wss{It is also possible to use \LaTeX for hypperlinks to external documents.}

\subsection{Module}

\wss{Short name for the module}

\subsection{Uses}

Fuzzy Logic Module (Section \ref{mFuzzyLogic})

\subsection{Syntax}

% \subsubsection{Exported Constants}

\subsubsection{Exported Access Programs}

\begin{center}
\begin{tabular}{p{2cm} p{4cm} p{4cm} p{2cm}}
\hline
\textbf{Name} & \textbf{In} & \textbf{Out} & \textbf{Exceptions} \\
\hline
decide & World & $\mathbb{R}$ & - \\
\hline
\end{tabular}
\end{center}

\subsection{Semantics}

\subsubsection{State Variables}

None
\wss{Not all modules will have state variables.  State variables give the module
  a memory.}

\subsubsection{Environment Variables}

None
\wss{This section is not necessary for all modules.  Its purpose is to capture
  when the module has external interaction with the environment, such as for a
  device driver, screen interface, keyboard, file, etc.}

\subsubsection{Assumptions}

None
\wss{Try to minimize assumptions and anticipate programmer errors via
  exceptions, but for practical purposes assumptions are sometimes appropriate.}

\subsubsection{Access Routine Semantics}
Because this module is interface so access routines doesn't have exact procedure.

\noindent decide(world):
\begin{itemize}
\item transition: -
\item output: Based on the current world situation, it should return proper 
      force value to keep the pendulum up. 
\item exception: -
\end{itemize}

\wss{A module without environment variables or state variables is unlikely to
  have a state transition.  In this case a state transition can only occur if
  the module is changing the state of another module.}

\wss{Modules rarely have both a transition and an output.  In most cases you
  will have one or the other.}

% \subsubsection{Local Functions}

% \wss{As appropriate} \wss{These functions are for the purpose of specification.
%   They are not necessarily something that is going to be implemented
%   explicitly.  Even if they are implemented, they are not exported; they only
%   have local scope.}

\newpage

\section{MIS of GUI Module} \label{mGUI} \wss{Use labels for
  cross-referencing}

\wss{You can reference SRS labels, such as R\ref{R_Inputs}.}

\wss{It is also possible to use \LaTeX for hypperlinks to external documents.}

\subsection{Module}

\wss{Short name for the module}
GUI

\subsection{Uses}
World Module (Section \ref{mWorld})

\subsection{Syntax}

% \subsubsection{Exported Constants}

\subsubsection{Exported Access Programs}

\begin{center}
\begin{tabular}{p{2cm} p{4cm} p{4cm} p{2cm}}
\hline
\textbf{Name} & \textbf{In} & \textbf{Out} & \textbf{Exceptions} \\
\hline
draw & World & - & - \\
\hline
\end{tabular}
\end{center}

\subsection{Semantics}

\subsubsection{State Variables}
None

\wss{Not all modules will have state variables.  State variables give the module
  a memory.}

\subsubsection{Environment Variables}

win: 2D diagram displayed on the screen

\wss{This section is not necessary for all modules.  Its purpose is to capture
  when the module has external interaction with the environment, such as for a
  device driver, screen interface, keyboard, file, etc.}

\subsubsection{Assumptions}

None
\wss{Try to minimize assumptions and anticipate programmer errors via
  exceptions, but for practical purposes assumptions are sometimes appropriate.}

\subsubsection{Access Routine Semantics}

\noindent draw(world):
\begin{itemize}
\item transition: Modify the win to display the current world state. The location 
      and angle of pendulum should be clear.
\item output: -
\item exception: -
\end{itemize}

\wss{A module without environment variables or state variables is unlikely to
  have a state transition.  In this case a state transition can only occur if
  the module is changing the state of another module.}

\wss{Modules rarely have both a transition and an output.  In most cases you
  will have one or the other.}

% \subsubsection{Local Functions}

% \wss{As appropriate} \wss{These functions are for the purpose of specification.
%   They are not necessarily something that is going to be implemented
%   explicitly.  Even if they are implemented, they are not exported; they only
%   have local scope.}

\newpage

% \section{MIS of ODE Solver Module} \label{mODESolver} \wss{Use labels for
%   cross-referencing}

% \wss{You can reference SRS labels, such as R\ref{R_Inputs}.}

% \wss{It is also possible to use \LaTeX for hypperlinks to external documents.}

% \subsection{Module}

% \wss{Short name for the module}

% \subsection{Uses}


% \subsection{Syntax}

% \subsubsection{Exported Constants}

% \subsubsection{Exported Access Programs}

% \begin{center}
% \begin{tabular}{p{2cm} p{4cm} p{4cm} p{2cm}}
% \hline
% \textbf{Name} & \textbf{In} & \textbf{Out} & \textbf{Exceptions} \\
% \hline
% \wss{accessProg} & - & - & - \\
% \hline
% \end{tabular}
% \end{center}

% \subsection{Semantics}

% \subsubsection{State Variables}

% \wss{Not all modules will have state variables.  State variables give the module
%   a memory.}

% \subsubsection{Environment Variables}

% \wss{This section is not necessary for all modules.  Its purpose is to capture
%   when the module has external interaction with the environment, such as for a
%   device driver, screen interface, keyboard, file, etc.}

% \subsubsection{Assumptions}

% \wss{Try to minimize assumptions and anticipate programmer errors via
%   exceptions, but for practical purposes assumptions are sometimes appropriate.}

% \subsubsection{Access Routine Semantics}

% \noindent \wss{accessProg}():
% \begin{itemize}
% \item transition: \wss{if appropriate} 
% \item output: \wss{if appropriate} 
% \item exception: \wss{if appropriate} 
% \end{itemize}

% \wss{A module without environment variables or state variables is unlikely to
%   have a state transition.  In this case a state transition can only occur if
%   the module is changing the state of another module.}

% \wss{Modules rarely have both a transition and an output.  In most cases you
%   will have one or the other.}

% \subsubsection{Local Functions}

% \wss{As appropriate} \wss{These functions are for the purpose of specification.
%   They are not necessarily something that is going to be implemented
%   explicitly.  Even if they are implemented, they are not exported; they only
%   have local scope.}

% \newpage

\bibliographystyle {plainnat}
\bibliography {../../../refs/References}

% \newpage

% \section{Appendix} \label{Appendix}

% \wss{Extra information if required}

% \section{Reflection}

% The information in this section will be used to evaluate the team members on the
% graduate attribute of Problem Analysis and Design.  Please answer the following questions:

% \begin{enumerate}
%   \item What are the limitations of your solution?  Put another way, given
%   unlimited resources, what could you do to make the project better? (LO\_ProbSolutions)
%   \item Give a brief overview of other design solutions you considered.  What
%   are the benefits and tradeoffs of those other designs compared with the chosen
%   design?  From all the potential options, why did you select the documented design?
%   (LO\_Explores)
% \end{enumerate}


\end{document}